\chapter{Imputation of missing values}
For various reasons, many real world datasets contain missing values, often encoded as blanks, NaNs or other placeholders. Such datasets however are incompatible with scikit-learn estimators which assume that all values in an array are numerical, and that all have and hold meaning. A basic strategy to use incomplete datasets is to discard entire rows and/or columns containing missing values. However, this comes at the price of losing data which may be valuable (even though incomplete). A better strategy is to impute the missing values, i.e., to infer them from the known part of the data. See the glossary entry on imputation.
\section{Univariate vs. Multivariate Imputation}
One type of imputation algorithm is univariate, which imputes values in the i-th feature dimension using only non-missing values in that feature dimension (e.g. impute.SimpleImputer). By contrast, multivariate imputation algorithms use the entire set of available feature dimensions to estimate the missing values (e.g. impute.IterativeImputer).