\chapter{Supervised learning}
\section{Linear Models\label{Linear Models}}
The following are a set of methods intended for regression in which the target value is expected to be a linear combination of the features. In mathematical notation, if $\hat{y}$ is the predicted value.
\begin{equation}
\hat{y}(w, x) = w_0 + w_1 x_1 + ... + w_p x_p
\end{equation}

Across the module, we designate the vector  $w = (w_1,
..., w_p)$ as \verb|coef_| and  $w_0$ as \verb|intercept_|.

To perform classification with generalized linear models, see \nameref{Logistic regression}.
\subsection{Ordinary Least Squares\label{Ordinary Least Squares}}
LinearRegression fits a linear model with coefficients $w = (w_1, ..., w_p)$ to minimize the residual sum of squares between the observed targets in the dataset, and the targets predicted by the linear approximation. Mathematically it solves a problem of the form:
\begin{equation}
\min_{w} || X w - y||_2^2
\end{equation}

LinearRegression will take in its \verb|fit| method arrays X, y and will store the coefficients $w$ of the linear model in its \verb|coef_| member:
\begin{minted}{python}
from sklearn import linear_model
reg = linear_model.LinearRegression()
reg.fit([[0, 0], [1, 1], [2, 2]], [0, 1, 2])
reg.coef_
\end{minted}
The coefficient estimates for Ordinary Least Squares \textbf{rely on the independence of the features}. When features are correlated and the columns of the design matrix $X$ have an approximately linear dependence, the design matrix becomes close to singular and as a result, the least-squares estimate becomes highly sensitive to random errors in the observed target, producing a large variance. This situation of multicollinearity can arise, for example, when data are collected without an experimental design.

\subsubsection{Examples:}
\begin{itemize}
\item \nameref{Linear Regression Example}
\end{itemize}
\subsubsection{Non-Negative Least Squares}
It is possible to constrain all the coefficients to be non-negative, which may be useful when they represent some physical or naturally non-negative quantities (e.g., frequency counts or prices of goods). LinearRegression accepts a boolean \verb|positive| parameter: when set to \verb|True| Non-Negative Least Squares are then applied.
\subsubsection{Examples:}
\begin{itemize}
\item \nameref{Non-negative least squares}
\end{itemize}

\subsubsection{Ordinary Least Squares Complexity}
The least squares solution is computed using the singular value decomposition of X. If X is a matrix of shape \verb|(n_samples, n_features)| this method has a cost of $O(n_{\text{samples}} n_{\text{features}}^2)$, assuming that $n_{\text{samples}} \geq n_{\text{features}}$.

\subsection{Ridge regression and classification}
\subsubsection{Regression}
Ridge regression addresses some of the problems of \nameref{Ordinary Least Squares} \textbf{by imposing a penalty on the size of the coefficients}. The ridge coefficients minimize a penalized residual sum of squares:
\begin{equation}
\min_{w} || X w - y||_2^2 + \alpha ||w||_2^2
\end{equation}

The complexity parameter $\alpha\ge 0$ controls the amount of shrinkage: the larger the value of $\alpha$, the greater the amount of shrinkage and thus the coefficients become more robust to collinearity.

\subsubsection{Classification}

The Ridge regressor has a classifier variant: RidgeClassifier. This classifier first converts binary targets to \verb|{-1, 1}| and then treats the problem as a regression task, optimizing the same objective as above. The predicted class corresponds to the sign of the regressor’s prediction. For multiclass classification, the problem is treated as multi-output regression, and the predicted class corresponds to the output with the highest value.

It might seem questionable to use a (penalized) Least Squares loss to fit a classification model instead of the more traditional logistic or hinge losses. However, in practice, all those models can lead to similar cross-validation scores in terms of accuracy or precision/recall, while the penalized least squares loss used by the RidgeClassifier allows for a very different choice of the numerical solvers with distinct computational performance profiles.

The RidgeClassifier can be significantly faster than e.g. LogisticRegression with a high number of classes because it can compute the projection matrix $(X^T X)^{-1} X^T$ only once.

This classifier is sometimes referred to as a \href{https://en.wikipedia.org/wiki/Least-squares_support_vector_machine}{Least Squares Support Vector Machines} with a linear kernel.
\subsubsection{Example}
\begin{itemize}
\item \nameref{Plot Ridge coefficients as a function of the regularization}
\item \nameref{Classification of text documents using sparse features}

\item \href{https://en.wikipedia.org/wiki/Hilbert_matrix}{Hilbert matrix}

\end{itemize}
\subsection{Lasso}
\subsection{Multi-task Lasso}

\subsection{Logistic regression\label{Logistic regression}}




